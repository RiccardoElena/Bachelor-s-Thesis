\chapter{Inference systems}\label{chap:inference-systems}

In chapter~\ref{chap:first-order-logic} we introduced FOL as a formal representation of knowledge, and we discussed its syntax and semantics.
Representation of knowledge alone is not sufficient for formalizing reasoning processes.
What is needed is a formalization of some reasoning mechanisms that can operate on this knowledge, rules that, given a set of premises, can derive new conclusions.

These rules are called \bold{inference rules}.
There are various types of inference rules, each with its own characteristics and applications.

In general, we denote with \(R\) a set of inference rules called \bold{inference system}. Given a set of premises \(\Gamma\), an inference rule \(r \in R\) can be applied to \(\Gamma\) to infer a new conclusion \(\phi\), denoted as \(\Gamma \vdash_r \phi\)\footnote{
  The notation can be simplified with just \(\Gamma \vdash \phi\) if the specific rule \(r\) is unambiguous.}.
If a sentence \(\phi\) can be inferred from the empty set using \(r\), we write \(\vdash_r \phi\).
It is possible to generalize this notation to the entire inference system \(R\) by writing \(\Gamma \vdash_{R} \phi\).

Moreover, it is possible to characterize the effectiveness of inference systems with various properties:

\begin{itemize}
    \item \bold{Soundness}: An inference system is sound if it only derives conclusions that are true in the model. Formally, \(R\) is \bold{sound} if for all \(\Gamma\) and \(\phi\), if \(\Gamma \vdash_{R} \phi\), then \(\Gamma\models\phi\).
    \item \bold{Completeness}: An inference system is complete if it can derive all conclusions that are true in the model. Formally, \(R\) is \bold{complete} if for all \(\Gamma\) and \(\phi\), if \(\Gamma\models\phi\), then \(\Gamma \vdash_{R} \phi\).
    \item \bold{Correctness}: An inference system is correct if it is both \bold{sound} and \bold{complete}.
    \item \bold{Consistency}: An inference system is consistent if it does not derive contradictory conclusions from any set of premises. Formally, \(R\) is \bold{consistent} if for all \(\Gamma\) and \(\phi\), it is not the case that both \(\Gamma \vdash_{R} \phi\) and \(\Gamma \vdash_{R} \neg \phi\).
\end{itemize}
A weaker form of completeness is \bold{refutational completeness} which requires that if \(\Gamma \models \bot\), then \(\Gamma \vdash_{R} \bot\).  Practically, this means that if a set of premises is unsatisfiable, the inference system can derive a contradiction.
By theorem~\ref{thm:validity_satisfiability_reduction}, a refutational complete inference system can show that a formula is valid by proving that its negation is unsatisfiable.

A first example of an inference rule is \bold{Modus Ponens}, which states that if we have a conditional statement \(A \implies B\), and we know that \(A\) is true, then we can conclude that \(B\) is also true. Formally, it can be expressed as:
\begin{equation}  
  \infer{B}{A & A \implies B}
\end{equation}


Or, alternatively, and more conveniently for formulae in CNF, we can express it as:
\begin{equation}  
  \infer{B}{A & \neg A \lor B}
\end{equation}

A classical application of Modus Ponens can be found in Aristotelian syllogisms, where we have two premises and a conclusion.
An example of a syllogism is:
\begin{quote}
  All humans are mortal. Socrates is a human. Therefore, Socrates is mortal.
\end{quote}
The premises can be formalized as:
\begin{equation}
  \forall x \ms H(x) \implies M(x), \qquad H(s)
\end{equation}
These can be converted into CNF as:
\begin{equation}
    \neg H(x) \lor M(x), \qquad H(s)
\end{equation}

In FOL, before applying Modus Ponens, we must ensure that the premises contain a matching literal. To do this, we need to find the MGU \(\sigma\) that makes the literals match that, in this case, is \(\sigma = \{(x,s)\}\).
Finally, we can apply Modus Ponens for deriving the conclusion \(M(s)\) (Socrates is mortal) as follows:
\begin{equation}
  \infer{M(s)}{H(s) & \neg H(x) \lor M(x)}
\end{equation}

From this example, we can define a more specific form of Modus Ponens for FOL, which takes into account the need for matching literals and the use of MGUs.
\begin{equation}  
  \infer{B\sigma}{A_1 & A_2 \implies B} \qquad \text{or} \qquad \infer{B\sigma}{A_1 & \neg A_2 \lor B}
\end{equation}
where \(\sigma\) is the MGU of \(A_1\) and \(A_2\).

While being a powerful inference rule, Modus Ponens has its limitations in the context of FOL\@.
In particular:
\begin{itemize}
  \item It is \bold{sound} because any conclusion derived is logically entailed.
  \item It is \bold{not complete} because there are valid conclusions that cannot be derived using this rule alone.
  \item It is \bold{not refutationally complete} because there are unsatisfiable sets of clauses that cannot be shown to be unsatisfiable using this rule alone.
\end{itemize}

For example, the set of clauses
\begin{equation}\label{eq:example_unsat}  
  \{\{\neg A, B\},\{\neg A , \neg B\}, \{\neg C , A\}, \{A , C\}\}
\end{equation}


This set is unsatisfiable:
\begin{description}
  \item[Assuming \(A\) true] From clause 1 we have \(B\) true, and from clause 2 we have \(\neg B\) true, leading to a contradiction.
  \item[Assuming \(A\) false] From clause 3 we have \(\neg C\) true, and from clause 4 we have \(C\) true, leading to a contradiction.
\end{description}
However, Modus Ponens alone cannot derive this contradiction because there are no atomic clauses to trigger the inference.

This limitation of Modus Ponens motivates the need for more general inference rules that can handle cases where no unit clauses are available to trigger inferences.
The resolution rule addresses this limitation by allowing inferences between any clauses containing complementary literals, regardless of their structure.

\section{Resolution and Factoring}\label{sec:resolution-factoring}

A common inference system used in FOL is based on the \bold{resolution} rule, which allows for the derivation of new clauses from existing ones, with the addition of the \bold{factoring} rule, that allows for the simplification of clauses by removing redundant literals.

The key insight is that Modus Ponens requires one premise to be a unit clause (single literal) to match against a conditional.
Resolution generalizes this by allowing any two clauses containing literals that become complementary after unification to be combined, making it much more powerful for systematic proof search in clause form.

\subsection{Resolution}\label{subsec:resolution}

\bold{Resolution} is a rule of inference specifically designed for CNF\@. It allows for the derivation of new clauses by resolving two clauses that contain complementary literals.

Formally, resolution in FOL can be defined as:

\begin{equation}
  \infer{(B\lor C)\sigma}{A_1 \lor B & \neg A_2 \lor C}
\end{equation}
\indent where \(\sigma\) is the MGU of \(A_1\) and \(A_2\).

\noindent Intuitively, resolution is justifiable noticing that, for \(A_1 \lor B\) and \(A_2 \lor C\) to be true, at least one of \(A_1\) and \(B\) must be true, and at least one of \(A_2\) and \(C\) must be true.
Being \(A_1\) and \(A_2\) complementary, it is impossible for both to be true at the same time, so at least one of \(B\) and \(C\) must be true. Thus, we can conclude that \(B \lor C\) must be true.

From this consideration, we can see that resolution is \bold{sound} because it only derives conclusions that are logically entailed by the premises.
However, resolution is \bold{not complete} because there are valid conclusions that cannot be derived using this rule alone.
An example of this is the \emph{disjunction introduction} rule \(A \models A \lor B\), which allows for the introduction of a disjunction from a single clause.

The absence of complementary literals in the premises prevents resolution from deriving any conclusions, and therefore to be complete.

Resolution though is \bold{refutationally complete}, meaning that if a set of clauses is unsatisfiable, resolution can derive the empty clause~\cite{robinson1965}, denoted \(\varnothing\), representing a contradiction --- a clause with no literals that can never be satisfied.
Using resolution, we can easily show that the previous example set of clauses~\ref{eq:example_unsat} is unsatisfiable:
\begin{equation}
  \begin{aligned}
    \neg A \lor B &\quad \text{Clause 1 (Premise)} \\
    \neg A \lor \neg B &\quad \text{Clause 2 (Premise)} \\
    \neg C \lor A &\quad \text{Clause 3 (Premise)} \\
    A \lor C &\quad \text{Clause 4 (Premise)} \\
    \neg A &\quad \text{Clause 5 (Resolution 1,2)} \\
    A &\quad \text{Clause 6 (Resolution 3,4)} \\
    \varnothing &\quad \text{Clause 7 (Resolution 5,6)}
  \end{aligned}
\end{equation}
\uninatodo{Not sure if this is the correct place for saturation algorithm. Maybe the vampire chapter is better suited?}
This can be achieved by \bold{saturating} the set of clauses. A set of clauses \(S\) is \bold{saturated} under an inference system \(R\) if all the conclusions that can be derived from premises in \(S\) using inferences in \(R\) are also included in \(S\) (up to redundancy).

If the saturation of a set of clause \(S\) does contain \(\varnothing\), then \(S\) is unsatisfiable.

\subsection{Factoring}\label{subsec:factoring}

In addition to resolution, \bold{factoring} is a rule that allows for the simplification of clauses by removing redundant literals.
In contrast to the inference rules seen so far, factoring does not derive new conclusions but rather simplifies existing clauses, improving their efficiency in the proof search.

Formally, factoring can be defined as:
\begin{equation}
  \infer{(C \lor A_1)\sigma}{C \lor A_1 \lor A_2}
\end{equation}
\indent where \(\sigma\) is the MGU of \(A_1\) and \(A_2\).

\noindent Factoring eliminates duplicate literals that arise after unification. If a clause contains two literals that can be unified with MGU \(\sigma\), then applying \(\sigma\) would make them identical, and we can simplify using the tautology \(A \lor A \iff A\).

For example, from the clause \(P(x) \lor Q(y) \lor P(f(y))\), we can factor on \(P(x)\) and \(P(f(y))\) with MGU \(\sigma = \{(x , f(y))\}\) to get \((P(x) \lor Q(y))\sigma = P(f(y)) \lor Q(y)\).

While resolution alone is refutationally complete, factoring serves as an optimization that reduces clause size and prevents the search space from being cluttered with subsumed clauses. This can significantly improve the performance of resolution-based theorem provers.

For instance, without factoring, we might derive increasingly complex clauses when the simpler factored form would suffice for the same logical content.

\section{Ordered Inferences}\label{sec:ordered-inferences}

While resolution and factoring provide a refutationally complete inference system, unrestricted application of these rules can lead to an explosion of irrelevant clauses, severely impacting efficiency. The key insight is that many inferences are redundant or unnecessary for finding a refutation.

Ordered inference rules address this problem by imposing systematic restrictions on when resolution and factoring can be applied, dramatically reducing the search space while preserving refutational completeness allowing inferences only on \emph{eligible} literals.

\subsection{Admissible Ordering}\label{subsec:admissible-ordering}

The foundation of ordered inferences is an \bold{admissible ordering} on literals, which guides the selection of inference candidates.

An ordering \(\succ\) on literals is \bold{admissible} if it satisfies:
\begin{itemize}
    \item \bold{Well-foundedness and totality on ground literals}: for any two ground literals \(L_1\) and \(L_2\), either \(L_1 \succ L_2\), \(L_2 \succ L_1\), or \(L_1 = L_2\) (totality), and each subset of ground literals has a minimal element (well-foundedness).
    \item \bold{Atom constraints}: for any atoms \(A\) and \(B\), it satisfies \(\neg A \succ A\) and \(B \succ A\) implies \(B \succ \neg A\).
    \item \bold{Stability under substitution (Liftable)}: if \(L_1 \succ L_2\), then \(L_1\sigma \succ L_2\sigma\) for any substitution \(\sigma\)
\end{itemize}

Common examples of admissible orderings include lexicographic orderings on terms and literals, where terms are compared first by their structure (function symbols, arity) and then by their arguments recursively.

For instance, using a lexicographic ordering where \(f \succ g\) and \(a \succ b\):
\[f(a) \succ f(b) \succ g(a) \succ g(b) \succ a \succ b\]

The ordering extends to literals by considering both the predicate symbols and their arguments, considering the \emph{depth} of terms.

Generally, a selection function is used to choose the most relevant literals for inference based on the admissible ordering; most of the time it is the function inducted from the ordering, selecting the maximal literals.

\subsection{Ordered Resolution}\label{subsec:ordered-resolution}

\bold{Ordered resolution} restricts the resolution rule to apply only when certain maximality conditions are satisfied.

Formally, ordered resolution can be defined as:

\begin{equation}
  \infer{(B\lor C)\sigma}{C\lor A_1 & \neg A_2 \lor B}
\end{equation}
  
\indent provided (i) \(\sigma\) is the MGU of \(A_1\) and \(A_2\), (ii) \(A_1\sigma\) is strictly maximal respect to \(B\sigma\), and \indent (iii) \(\neg A_2\) is maximal with respect to \(D\sigma\).



\noindent The intuition is that we only resolve on the \dquote{largest} literals in each clause according to our ordering. This prevents resolution on smaller literals that might lead to unnecessarily complex derived clauses.

Consider the clauses:
\[P(f(x)) \lor Q(x), \qquad \neg P(y) \lor R(y)\]

If our ordering satisfies \(P(f(x)) \succ Q(x)\) and \(P(y) \succ R(y)\), then ordered resolution can proceed on the \(P\) literals, yielding:
\[Q(x) \lor R(f(x))\]

However, if \(Q(x) \succ P(f(x))\), then ordered resolution would be blocked, as \(P(f(x))\) is not maximal in the first clause.

While maintaining refutational completeness, ordered resolution, in practice, can reduce the number of generated clauses by several orders of magnitude, making automated theorem proving feasible for larger problems.

Moreover, using ordered resolution in a saturation process, can lead to termination on inputs that standard resolution cannot handle.

\subsection{Ordered Factoring}\label{subsec:ordered-factoring}

Similar restrictions apply to factoring to maintain consistency with ordered resolution.

It is defined as:
\begin{equation}
  \infer{(C \lor A_1)\sigma}{C \lor A_1 \lor A_2}
\end{equation}
\indent provided (i) \(\sigma\) is the MGU of \(A_1\) and \(A_2\), and (ii) \(A_1\sigma\) is maximal with respect to \(C\sigma\).

\noindent For example, consider the clause:
\[P(f(x)) \lor Q(y) \lor P(x)\]

If \(P(f(x)) \succ P(x) \succ Q(y)\) and we can unify \(P(f(x))\) and \(P(x)\) with \(\sigma = \{(x , f(x))\}\), then ordered factoring produces:
\[P(f(x)) \lor Q(y)\]

The combination of ordered resolution and ordered factoring creates a highly efficient inference system that maintains the theoretical guarantees of refutational completeness while achieving practical performance suitable for real-world theorem proving applications.

The ordering strategies are particularly effective because they tend to eliminate smaller, more general literals first, leading toward the specific contradictions needed for refutation more directly than unrestricted search.


